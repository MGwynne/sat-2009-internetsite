%%% sat09-cfp.tex --- 

% Original author: Joao Marques-Silva
% Adapted by Oliver Kullmann, 22.10.2008.

%%%%##########################################################################
%%%% Configurations
%%%%
\documentclass[10pt]{article}

\usepackage[usenames]{color}
\usepackage{url}
\usepackage{hyperref}

\usepackage{a4wide}


%% Redefine the page layout
%%
\addtolength{\textheight}{60pt}
\addtolength{\topmargin}{-60pt}
\addtolength{\textwidth}{100pt}
\addtolength{\oddsidemargin}{-50pt}
\addtolength{\evensidemargin}{-50pt}
\setlength{\marginparwidth}{0pt}

\def \baselinestretch{0.95}

\newcommand{\bthlight}[1]{{\color[rgb]{0.2,0.2,0.5}#1}}
\newcommand{\bhlight}[1]{{\color[rgb]{0.2,0.2,0.6}#1}}
\newcommand{\ghlight}[1]{{\color[rgb]{0.1,0.4,0.1}#1}}

\definecolor{URLGreen}{rgb}{0.1,0.5,0.1} 

\hypersetup{colorlinks=true,pdfborder={0 0 0},urlbordercolor={0 0 0},
urlcolor=URLGreen}
%linkcolor={0.1 0.4 0.1},filecolor={0.1 0.4 0.1}


%%%%##########################################################################
%%%% The main document
%%%%
\begin{document}

\pagestyle{empty}

\begin{center}
  {\bf
    \bthlight{{\huge SAT 2009}} \\[0.3cm]
    {\Large Call for Papers} \\[0.15cm]
    \bthlight{
      {\LARGE 12th International Conference on} \\[0.2cm]
      {\LARGE Theory and Applications of Satisfiability Testing} \\[0.15cm]
      {\Large June 30 - July 3, 2009, Swansea, Wales, United Kingdom}
    } \\[0.15cm]
    {\large \url{http://www.cs.swan.ac.uk/~csoliver/SAT2009}}
  }
\end{center}

\vspace*{-0.1cm}
%
\begin{minipage}[t]{7.5cm}
{\large {\bf \bhlight{Conference Chair}}} \\[0.05cm]
{\small
  \href{http://www.cs.swan.ac.uk/~csoliver/index.html}{Oliver Kullmann}, Swansea University, Swansea, UK\\[0.3cm]
}
{\large {\bf \bhlight{Invited Speakers}}} \\%[0.1cm]
{\small
  \href{http://www.lsi.upc.edu/~roberto/}{Robert Nieuwenhuis}, Tech.~Univ.~Catalonia, Spain\\
  \href{http://www.cs.rice.edu/~vardi/}{Moshe Vardi}, Rice University, U.S.A.\\[0.3cm]
}
{\large {\bf \bhlight{Important Dates}}} \\[0.05cm]
{\small
  February 20, {\em Abstract Submission} \\
  February 27, {\em Paper Submission} \\
  March 22, {\em Author Notification} \\
  March 29, {\em Final Version} \\[0.3cm]
}
{\large {\bf \bhlight{Technical Program Committee}}} \\[0.01cm]
{\footnotesize
          \href{http://www.cs.ucsc.edu/~optas/}{Dimitris Achlioptas}, UC Santa Cruz, USA \\
          \href{http://fmv.jku.at/biere/}{Armin Biere}, Johannes Kepler University, Austria  \\
          \href{http://www.cs.toronto.edu/~sacook/}{Stephen A.\ Cook}, University of Toronto, Canada \\
          \href{http://www.dil.univ-mrs.fr/membres.html}{Nadia Creignou}, Universite de la Mediterranee, France  \\
          \href{http://cs.roosevelt.edu/~dantsin/}{Evgeny Dantsin}, Roosevelt University, Chicago \\
          \href{http://www.cs.ucla.edu/~darwiche/}{Adnan Darwiche}, UCLA, USA \\
          \href{http://www.ececs.uc.edu/~franco/}{John Franco}, University of Cincinnati, USA  \\
          \href{http://www.dsi.uniroma1.it/~galesi/}{Nicola Galesi}, Universita di Roma, Italy  \\
          \href{http://www.star.dist.unige.it/index.php?option=com_uhp2&task=viewpage&item_id=50&user_id=74}{Enrico Giunchiglia}, Universita di Genova, Italy \\
          \href{http://www.st.ewi.tudelft.nl/~marijn/}{Marijn Heule}, Technische Universiteit Delft, Netherlands \\
          \href{http://logic.pdmi.ras.ru/~hirsch/}{Edward Hirsch}, Steklov Institute of Mathematics, Russia  \\
          \href{http://www.lab2.kuis.kyoto-u.ac.jp/~iwama/}{Kazuo Iwama}, Kyoto University, Japan  \\
          \href{http://www.cril.univ-artois.fr/~leberre/}{Daniel LeBerre}, Universite d'Artois, France  \\
          \href{http://www.laria.u-picardie.fr/~cli/}{Chumin Li}, Universite de Picardie, France  \\
          \href{http://sat.inesc-id.pt/~ines/}{Ines Lynce}, Instituto Superior Tecnico, Portugal \\
          \href{http://www.ccs.neu.edu/home/pete}{Panagiotis Manolios}, Northeastern University, USA  \\
          \href{http://users.ecs.soton.ac.uk/jpms/}{Joao Marques-Silva}, University of Southampton, UK  \\
          \href{http://www.cs.sfu.ca/~mitchell/}{David G.\ Mitchell}, Simon Fraser University, Canada  \\
          \href{http://wwwcs.uni-paderborn.de/cs/ag-klbue/en/staff/kbcsl/index.html}{Hans Kleine Buening}, University of Paderborn, Germany  \\
          \href{http://www.lsi.upc.edu/~oliveras/}{Albert Oliveras}, Technical University of Catalonia, Spain  \\
          \href{http://www-cse.ucsd.edu/~paturi/}{Ramamohan Paturi}, University of California, USA  \\
          \href{http://www.cril.univ-artois.fr/~sais/}{Lakhdar Sais}, Universite d'Artois, France  \\
          \href{http://www.eecs.umich.edu/~karem/}{Karem A.\ Sakallah}, University of Michigan, USA  \\
          \href{http://theorie.informatik.uni-ulm.de/Personen/Schoening/index.html}{Uwe Schoening}, Universitat Ulm, Germany   \\
          \href{http://www.dit.unitn.it/~rseba/}{Roberto Sebastiani}, Universita di Trento, Italy  \\
          \href{http://www.cs.uic.edu/~sloan/}{Robert H.\ Sloan}, University of Illinois, USA  \\
          \href{http://www.cs.chalmers.se/~nik/}{Niklas Sorensson}, Chalmers University of Tech, Sweden  \\
          \href{http://www.scale.uni-koeln.de/}{Ewald Speckenmeyer}, Universitat Koln, Germany  \\
          \href{http://www.dur.ac.uk/stefan.szeider/}{Stefan Szeider}, Durham University, UK  \\
          \href{http://www.mrg.dist.unige.it/~tac/}{Armando Tacchella}, Universita di Genova, UK  \\
          \href{http://www.soe.ucsc.edu/~avg/}{Allen Van Gelder}, UC Santa Cruz, USA  \\
          \href{http://www.st.ewi.tudelft.nl/~maaren/}{Hans van Maaren},  Delft University of Tech, Netherlands  \\
          \href{http://www.cse.unsw.edu.au/~tw/}{Toby Walsh}, University of New South Wales, Australia  \\
          \href{http://www.cs.uc.edu/~weaversa/}{Sean Weaver}, University of Cincinnati, USA  \\
          \href{http://www.inf.ethz.ch/personal/emo/}{Emo Welzl}, ETH Zurich, Switzerland  \\
          \href{http://research.microsoft.com/users/lintaoz/}{Lintao Zhang}, Microsoft Research, USA  \\
          \href{http://logic.sysu.edu.cn/2005/english/PEOPLE/200510/english_277.html}{Xishun Zhao}, Sun Yat-Sen University, P.R. China  \\
}
\end{minipage}
\hspace*{0.35cm}
\begin{minipage}[t]{9.25cm}
  \begin{minipage}[t]{9.25cm}
    \vspace*{0.15cm}
  {\small
  The International Conference on Theory and Applications of
  Satisfiability Testing (\href{http://www.satisfiability.org}{SAT})
  is the primary annual meeting for researchers studying the
  propositional satisfiability
  problem. \href{http://cs-svr1.swan.ac.uk/~csoliver/SAT2009}{SAT
    2009} is the twelfth SAT conference. SAT 2009 features the
  \href{http://www.satcompetition.org/2009}{SAT competition}, the
  \href{http://www.cril.univ-artois.fr/PB09}{Pseudo-Boolean evaluation}, and the
  \href{http://www.maxsat.udl.cat/09/}{MAX-SAT evaluation}. \\[0.1cm]
%
  Many hard combinatorial problems can be encoded into
  SAT. Therefore improvements on heuristics on the practical side, as
  well as theoretical insights into SAT, apply to a large range of
  real-world problems. More specifically, many important practical
  verification problems can be rephrased as SAT problems. This
  applies to verification problems in hardware and software. Thus SAT
  is becoming one of the most important core technologies to verify
  secure and dependable systems. The topics of the conference span
  practical and theoretical research on SAT and its applications and
  include but are not limited to proof systems, proof complexity,
  search algorithms, heuristics, analysis of algorithms, hard
  instances, randomised formulae, problem encodings, industrial
  applications, solvers, simplifiers, tools, case studies and
  empirical results. SAT is interpreted in a rather broad sense:
  besides propositional satisfiability, it includes the domain
  of quantified boolean formulae (QBF), constraints programming
  techniques (CSP) for word-level problems and their propositional
  encoding and particularly satisfiability modulo theories (SMT). \\[0.1cm]
%
  Submissions should contain original material and can either be
  regular research papers up to 14 pages or short papers up to 6
  pages. Double submissions including submissions as short and long
  papers will be rejected.  Submissions should use the 
  \href{http://www.springer.com/comp/lncs/Authors.html}{Springer LNCS}
  style. All appendices, tables, figures and the bibliography 
  must fit into the page limit. Submissions deviating from these
  requirements may be rejected without review. All accepted papers
  including short papers will be published in the proceedings of the
  conference. The conference proceedings will be published within
  the Springer LNCS series. The paper submission page is
  \url{http://www.easychair.org/conferences/?conf=sat2009}. Papers have to
  be submitted electronically as PDF files. Abstract submissions are due
  Februay 20, paper submissions February 27.
  }
  \end{minipage}

  \begin{minipage}[t]{9.25cm}
    \vspace*{0.35cm}
    \begin{minipage}[t]{4.5cm}
      \bthlight{{\bf SAT Competition}} \\
               {\small \url{www.satcompetition.org/2009}}\\[0.15cm]
               Daniel Le Berre\\
               Laurent Simon\\[0.15cm]
               Andreas Goerdt\\
               ??? \\
               ??? %\\ [0.15cm]
    \end{minipage}
    \hspace*{0.5cm}
    \begin{minipage}[t]{4.25cm}
      \bthlight{{\bf MAX-SAT Evaluation}}\\
               {\small \url{www.maxsat.udl.cat/09/}}\\[0.15cm]
               Josep Argelich\\
               Chu-Min Li\\
               Felip Many\`{a}\\
               Jordi Planes %\\[0.15cm]
    \end{minipage}

    \begin{minipage}[t]{9.25cm}
      \vspace*{0.25cm}
      \begin{minipage}[t]{6.0cm}
        \bthlight{{\bf PB Evaluation}}\\
                 {\small \url{http://www.cril.univ-artois.fr/PB09}}\\[0.15cm]
                 Vasco Manquinho\\
                 Olivier Roussel %\\[0.15cm]
      \end{minipage}
    \end{minipage}
  \end{minipage}


\end{minipage}
%

\end{document}

%%%%
%%%%##########################################################################

